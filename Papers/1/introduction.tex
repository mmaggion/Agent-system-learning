% !TEX root = dynamicslearning.tex
\section{Introduction}

What are the instinctive individual reactions which make a group of animals forming coordinated movements, for instance a flock of migrating birds or a school of fish? Which biological interactions between cells produce the formation of complex structures, for instance organs? What are the mechanisms which induce certain significant changes in a large amount of players in the financial market? 
In this paper we are concerned with the ``mathematization'' of the problem of learning or inferring interaction rules from observations of evolutions. The framework we consider is the one of  evolutions driven by gradient descents.
The study of gradient flow evolutions to minimize  certain energetic landscapes has been the subject of intensive research in the past years \cite{AGS}. Some of the most recent models are  aiming at describing time-dependent phenomena also in biology or even in social dynamics, borrowing a leaf from more established and classical  models in physics. For instance, starting with the seminal papers of Vicsek et. al. \cite{VCBCS95} and Cucker-Smale \cite{CucSma07}, there has been a flood of models describing consensus or opinion formation,  modeling the exchange of information as long-range social interactions (forces) between active agents (particles). However, for the analysis, but even more crucially for the reliable and realistic numerical simulation of such phenomena, one presupposes a complete understanding and determination of the governing energies. Unfortunately, except for physical situations where the calibration of the model can be done by measuring the governing forces rather precisely, for some relevant macroscopical models in physics and most of the models in biology and social sciences the governing energies are far from being precisely determined. In fact, very often in these studies the governing energies are just predetermined to be able to reproduce, at least approximately or qualitatively, some of the macroscopical effects of the observed dynamics, such as the formation of certain patterns, but there has been little or no effort of matching data from real-life cases. 




This attitude aiming just at a qualitative description tends however to reduce  some of the investigations in this area to  beautiful and mathematically interesting toy-cases, which have likely little to do with real-life scenarios. 
%We also have been directly involved in some of these developments  and we recognize by now certain significant limitations towards the realistic applicability of some of these models \cite{}.\\
The aim of this paper is providing a  mathematical framework for the reliable identification of the governing energies from data obtained by direct observations of corresponding time-dependent evolutions. This is a new kind of inverse problem, beyond more traditionally considered ones, as the forward map is a strongly nonlinear  evolution, highly dependent on the probability measure generating the initial conditions. As we aim at a precise quantitative analysis, and to be very concrete, we will  attack the learning of the energies for specific models in social dynamics governed by nonlocal interactions.


