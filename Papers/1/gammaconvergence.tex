% !TEX root = dynamicslearning.tex

\section{$\Gamma$-convergence of $\mathcal E_N$ to $\mathcal E$}

%At the end of the last section, we have seen which are the main ingredients to ensure that our error functional $E$ has the target interaction kernel $a$ as unique minimizer (the coercivity assumption \eqref{eq-coercive}), and for the well-posedness of the minimization problem of the functionals $E_N$ (the knowledge of an upper bound for $\|a\|_{L_{\infty}(\R_+)}$ and $\|a'\|_{L_{\infty}(\supp(\rho))}$).

This section is devoted to a proof of Theorem \ref{thm}.
We start with a technical lemma.

\begin{lemma}\label{lemma-semicontinuous-1}
	Under the assumptions of Theorem \ref{thm}, let $(b_N)_{N \in \N}\subset X_{M,K}$ be a sequence of continuous functions uniformly converging to a function $b \in X_{M,K}$ on $K=[0,2R]$  with $R>0$ as in \eqref{Rest}. % with the property that there exists an $M > 0$ such that $\|\widehat a_N\|_{L_{\infty}(\supp(\rho))} \leq M$ for every $N \in \N$.
Then it holds
	\begin{align*}
		\lim_{N\rightarrow\infty} \mathcal E_{N}(b_{N})= \mathcal E(b).
	\end{align*}
\end{lemma}

\begin{proof}
	From \cite[Lemma 3.3]{fornahuetter} it follows $\W_1(\mu^0,\mu_N^0) \rightarrow 0$ for $N \rightarrow \infty$. Hence, from \eqref{stab} we have that $W_1(\mu(t),\mu_N(t))\rightarrow 0$ for $N\rightarrow\infty$, uniformly for  $t \in [0,T]$. %, which in particular implies the weak convergence of the sequence of measures $(\mu_N(t))_{N \in \N}$ towards $\mu(t)$ for all $t\in [0,T]$.
Moreover, for all $x,y,y' \in B(0,R)$, by triangle inequality we have
\begin{align*}
|(\Fun{a} -\Fun{b})(x-y') - &(\Fun{a} -\Fun{b})(x- y)|  \\
& \leq \left[2R (\Lip_K(a) + \Lip_K( b))    +\|a\|_{L_\infty(K)} + \| b\|_{L_\infty(K)} \right] |y-y'|,
\end{align*}
which implies the uniform Lipschitz continuity of $(\Fun{a} -\Fun{b})(x- \cdot)$ in $B(0,R)$, for fixed $x\in B(0,R)$  \MMcomment{was:with respect to $x \in B(0,R)$}.
For every $\varepsilon > 0$ one can find $N_0(\varepsilon)$ such that, for all $N \geq N_0(\varepsilon)$ we have
	\begin{align*}
		\sup_{x,y \in B(0,R)}|\Fun{b_{N}}(x-y)-\Fun{b}(x-y)|
			\leq 2R \| b_{N}- b\|_{L_\infty(K)}\leq\varepsilon/2,
	\end{align*}
	as well as
	\begin{align*}
		\biggl|\int_{\R^d}\bigl(\Fun{b}-\Fun{a}\bigr)(x-y)d\mu^{N}(t)(y)
			-\int_{\R^d}\bigl(\Fun{b}-\Fun{a}\bigr)(x-y)d\mu(t)(y)\biggr|\leq\varepsilon/2,
	\end{align*}
uniformly with respect to $t \in [0,T]$ and $x \in B(0,R)$.
	The first estimate follows from  the uniform convergence of the $ b_{N}$, while the second one follows from the uniform Lipschitz continuity of
	$(\Fun{a} -\Fun{b})(x- \cdot)$ for fixed $x\in B(0,R)$ \MMcomment{was:with respect to $x \in B(0,R)$}  and the uniform Wasserstein convergence of $\mu^{N}(t)$ to $\mu(t)$ with respect to $t \in [0,T]$. Hence for $N\geq N_0(\varepsilon)$ we obtain
	\begin{align*}
		\Biggl|
			&\biggl|\int_{\R^d}\bigl(\Fun{b_{N}}-\Fun{a}\bigr)(x-y)d\mu^{N}(t)(y)\biggr|
				-\biggl|\int_{\R^d}\bigl(\Fun{b}-\Fun{a}\bigr)(x-y)d\mu(t)(y)\biggr|\Biggr|\\
			&\leq\biggl|\int_{\R^d}\bigl(\Fun{b_{N}}-\Fun{a}\bigr)(x-y)d\mu^{N}(t)(y)
					-\int_{\R^d}\bigl(\Fun{b}-\Fun{a}\bigr)(x-y)d\mu(t)(y)\biggr|\\
			&\leq\Biggl|\int_{\R^d}
				\bigl(\Fun{b_{N}}-\Fun{b}\bigr)(x-y)d\mu^{N}(t)(y)\Biggr|\\
			&\qquad +\biggl|\int_{\R^d}\bigl(\Fun{b}-\Fun{a}\bigr)(x-y)d\mu^{N}(t)(y)
					-\int_{\R^d}\bigl(\Fun{b}-\Fun{a}\bigr)(x-y)d\mu(t)(y)\biggr|\\
			&\leq2R \| b_{N}-\widehat  a\|_{L_\infty(K)}\int_{\R^d}d\mu^{N}(t)(y)+\frac{\varepsilon}{2}=\varepsilon\,.
	\end{align*}
	Therefore, for every $t \in [0,T]$ and $x \in B(0,R)$, 
	\begin{align}\label{firstlim}
		\lim_{N\rightarrow\infty}\Biggl|\int_{\R^d}\bigl(\Fun{b_{N}}-\Fun{a}\bigr)(x-y)d\mu^{N}(t)(y)\Biggr|^2
			=\Biggl|\int_{\R^d}\bigl(\Fun{b}-\Fun{a}\bigr)(x-y)d\mu(t)(y)\Biggr|^2.
	\end{align}
%	For $t \in [0,T]$, we now pass to compute the limit
%	\begin{align*}
%		\lim_{N\rightarrow\infty}
%			\int_{\R^d}\Biggl|\int_{\R^d}\bigl(\Fun{b_{N}}-\Fun{a}\bigr)(x-y)d\mu^{N}(t)(y)\Biggr|^2d\mu_N(t)(x).
%	\end{align*}
	Denote
	\begin{equation*}
	\begin{split}
		H_N(t,x)&=\Biggl|\int_{\R^d}\bigl(\Fun{b_{N}}-\Fun{a}\bigr)(x-y)d\mu^{N}(t)(y)\Biggr|^2\,, \\
		H(t,x)&= \Biggl|\int_{\R^d}\bigl(\Fun{b}-\Fun{a}\bigr)(x-y)d\mu(t)(y)\Biggr|^2\,,
		\end{split}
		\qquad
		\begin{split}
		G_N(t)&= \int_{\R^d}H_N(t,x)d\mu_N(t)(x)\,,\\
		G(t)&= \int_{\R^d}H(t,x)d\mu(t)(x),
		\end{split}	
	\end{equation*}
	and we estimate
	\begin{align}
		|G_N(t)-G(t)|&\leq \left|\int_{\R^d}H(t,x)d\mu_N(t)(x) - \int_{\R^d}H(t,x)d\mu(t)(x)\right| \nonumber\\
		&\quad + \int_{\R^d}\left|H_N(t,x) - H(t,x)\right|d\mu_N(t)(x). \label{pippo3}
	\end{align}
	We now prove that the function $H$ is Lipschitz continuous with respect to $x \in B(0,R)$ uniformly in $t \in [0,T]$. To do so, we write $H$ as
	\begin{align*}
	H(t,x) = g\left(\int_{\R^d} f(x,y) d\mu(t)(y)\right),
	\end{align*}
	where $g$ is a differentiable function in $\R^d$ and $f(\cdot,y)$ is a Lipschitz continuous function uniformly in $y$ with values in $\R^d$ (in this case $f(\cdot,y)=(\Fun{a} -\Fun{b})(\cdot -y)$). Since the measure $\mu(t)$ has support contained in $B(0,R)$, it follows
	\begin{align*}
	\left|\int_{\R^d} f(x,y) d\mu(t)(y)\right| &\leq \int_{\R^d} |f(x,y)| d\mu(t)(y) \leq \sup_{x,y \in B(0,R)} |f(x,y)| =: S <+\infty.
	\end{align*}
	Therefore, for every $x, x' \in B(0,R)$
	\begin{align*}
	\left|H(t,x)-H(t,x')\right| &\leq \Lip_{B(0,S)} (g) \left|\int_{\R^d} f(x,y) d\mu(t)(y) - \int_{\R^d} f(x',y) d\mu(t)(y)\right| \\
	&\leq \Lip_{B(0,S)} (g) \int_{\R^d} \left|f(x,y)  - f(x',y)\right|  d\mu(t)(y)\\
	&\leq \Lip_{B(0,S)} (g) \; \Lip_{B(0,R)} (f) \; |x-x'|,
	\end{align*}
	from which follows the Lipschitz continuity of $H(t,x)$ with respect to $x \in B(0,R)$ uniformly in $t \in [0,T]$.
	
	From this uniform Lipschitz continuity and the uniform Wasserstein convergence of $\mu_N(t)$ to $\mu(t)$ with respect to $t \in [0,T]$, it follows that for every $\varepsilon > 0$ we can find $N_0(\varepsilon)$ such that for all $N \geq N_0(\varepsilon)$ it holds
	\begin{align}\label{pippo1}
		\left|\int_{\R^d}H(t,x)d\mu_N(t)(x) - \int_{\R^d}H(t,x)d\mu(t)(x)\right|\leq \frac{\varepsilon}{2},
	\end{align}
	uniformly with respect to $t \in [0,T]$. From \eqref{firstlim} it follows also that for all $N \geq N_0(\varepsilon)$ we have
	\begin{align}\label{pippo2}
	\left|H_N(t,x) - H(t,x)\right| \leq \frac{\varepsilon}{2},
	\end{align}
	uniformly with respect to $t \in [0,T]$ and $x \in B(0,R)$. A combination of \eqref{pippo3} with \eqref{pippo1} and \eqref{pippo2}  yields $|G_N(t)-G(t)| \leq \varepsilon$ uniformly in $t \in [0,T]$. Hence
	\begin{align*}
		\lim_{N\rightarrow\infty}
			\int_{\R^d}\Biggl|\int_{\R^d}\bigl(\Fun{b_{N}}-&\Fun{a}\bigr)(x-y)d\mu^{N}(t)(y)\Biggr|^2d\mu_N(t)(x) = \\
			&\int_{\R^d}\Biggl|\int_{\R^d}\bigl(\Fun{b}-\Fun{a}\bigr)(x-y)d\mu(t)(y)\Biggr|^2d\mu(t)(x),
	\end{align*}
	 holds uniformly in $t \in [0,T]$ \MMcomment{does this just mean ``holds for every $t\in[0,T]$''?}.
	
	To eventually show that $\lim_{N \rightarrow \infty} \mathcal E_N( b_N) = \mathcal E( b)$, we simply note that uniform convergence of $G_N$ to $G$ implies 
	\begin{align*}
		\lim_{N \rightarrow \infty} \mathcal E_N( b_N) =  \lim_{N\rightarrow\infty} \frac{1}{T} \int^T_0 G_N(t) dt = \frac{1}{T} \int^T_0 G(t) dt = \mathcal E( b).
	\end{align*}
%	But this follows easily from the dominated convergence theorem and the fact that $T$ is finite, since we can bound the functions $G_N$ uniformly from above using \eqref{Faest} as
%	\begin{align*}
%		|G_N(t)| & \leq \int_{\R^d}\Biggl(\int_{\R^d}\bigl|(F[\widehat a_{N}]-\Fun{a})(x-y)\bigr|d\mu^{N}(t)(y)\Biggr)^2d\mu_N(t)(x) \\
%		& \leq 4R^2 \|\widehat a_N - a \|^2_{L_{\infty}(K)} \\
%		& \leq 4R^2(B + \|a \|_{L_{\infty}(K)})^2.
%	\end{align*}
%	Here $B$ is an appropriate bound for $\|\widehat a_N\|^2_{L_{\infty}(K)}$, which exists since $(\widehat a_N)_{N \in\N}$ is a uniformly convergent sequence on $K$ (and thus bounded).
%	Thus applying the result from {\it arXiv-article}, we obtain
%	\begin{align*}
%		\liminf_k\frac{1}{T}\int_0^T\int_{\R^d}
%			&\Bigl\|\bigl(F[a_{N_k}]-\Fun{a}\bigr)\ast\mu_t^{N_k}(x)\Bigr\|^2 d\mu_t^{N_k}(x) dt\\
%			&\geq\frac{1}{T}\int_0^T \int_{\R^d} \liminf_{\substack{k\rightarrow\infty,\\x'\rightarrow x}}
%				\Bigl\|\bigl(F[a_{N_k}]-\Fun{a}\bigr)\ast\mu_t^{N_k}(x')\Bigr\|^2 d\mu_t(x) dt\\
%			&=\frac{1}{T}\int_0^T\int_{\R^d}\liminf_{k\rightarrow\infty}
%				\Bigl\|\bigl(F[a_{N_k}]-\Fun{a}\bigr)\ast\mu_t^{N_k}(x)\Bigr\|^2 d\mu_t(x) dt\,,\\
%	\end{align*}
%	where the last line is due to continuity of the function
%	\begin{align*}
%		x'\mapsto\Bigl\|\bigl(F[a_{N_k}]-\Fun{a}\bigr)\ast\mu_t^{N_k}(x')\Bigr\|\,.
%	\end{align*}
\end{proof}

%\begin{lemma}\label{lemma-semicontinuous-2a}
%	Let $(a_N)_N\subset L_2(\R_+,\rho)$ be a sequence of continuous functions which converges pointwise
%	$\cl_1$-almost everywhere to some function $\overline a$. Then for every $\varepsilon>0$ there exist sets
%	$K^t_\varepsilon\subset\R^d$, $t\in [0,T]$, such that $\mu_t(\R^d\setminus K^t_\varepsilon)\leq\varepsilon$ and
%	\begin{equation}\label{eq-lower-semi-2}
%		\begin{split}
%		\liminf_{N\rightarrow\infty}&E_N(a_N)\\
%			&\geq\frac{1}{T}\int_0^T\int_{\R^d}\Biggl(\Biggl\|\int_{K^t_\varepsilon}
%				\bigl(F[\bar a]-\Fun{a}\bigr)(x-y)d\mu_t(y)\Biggr\|
%				-4\varepsilon\|a\|_\infty\Biggr )^2 d\mu_t(x)\,.
%		\end{split}
%	\end{equation}
%\end{lemma}
%
%\begin{proof}
%	{\bf Step 1:}
%	Step 2 of the proof of Lemma \ref{lemma-semicontinuous-1} can be re-used here.
%	
%	Since $\supp\mu_t^N\subset A$ (compact) uniformly w.r.t. $N$ and $t$ {\it (reference)} by Egorov's
%	Theorem for all $\varepsilon>0$ and all $t>0$ there exist measurable sets $K_\varepsilon^t\subset A$ with
%	$\mu_t(A\setminus K_\varepsilon^t)\leq\varepsilon$ such that
%	$F[\widehat a_N]\rightrightarrows F[\overline a]$ uniformly on $K_\varepsilon^t$. Splitting the integration
%	domain accordingly and using triangle inequality we find
%	\begin{align*}
%		\Biggl\|
%			&\int_{\R^d}\bigl(F[a_N]-\Fun{a}\bigr)(x-y)d\mu_t^N(y)\Biggr\|\\
%			&=\Biggl\|\int_{K_\varepsilon^t}\bigl(F[a_N]-\Fun{a}\bigr)(x-y)d\mu_t^N(y)
%				+\int_{A\setminus K_\varepsilon^t}\bigl(F[a_N]-\Fun{a}\bigr)(x-y)d\mu_t^N(y)\Biggr\|\\
%			&\geq\underbrace{\Biggl\|\int_{K_\varepsilon^t}
%				\bigl(F[a_N]-\Fun{a}\bigr)(x-y)d\mu_t^N(y)\Biggr\|}_{I_N}
%				-\underbrace{\Biggl\|\int_{A\setminus K_\varepsilon^t}
%					\bigl(F[a_N]-\Fun{a}\bigr)(x-y)d\mu_t^N(y)\Biggr\|}_{II_N}\,.
%	\end{align*}
%	
%	For the term $I_N$ we further find by combining the uniform convergence of $a_N$ on
%	$K_\varepsilon^t$ and the weak convergence of $\mu_t^N$ that the limit $N\rightarrow\infty$ exists: For
%	$N\geq N_0(\delta)$ we have
%	\begin{equation*}
%		\|F[a_N]-F[\overline a]\|
%			=\|a_N-\overline a\|_{L_\infty(K_\varepsilon^t)}\leq\delta/2\,,
%	\end{equation*}
%	as well as
%	\begin{equation*}
%		\biggl\|\int_{K_\varepsilon^t}\bigl(F[\overline a]-\Fun{a}\bigr)(x-y)d\mu_N_t(y)
%			-\int_{K_\varepsilon^t}\bigl(F[\overline a]-\Fun{a}\bigr)(x-y)d\mu_t(y)\biggr\|\leq\delta/2\,,
%	\end{equation*}
%	note that continuity of $a$ and $\overline a$ (the latter holds only on $K_\varepsilon^t$!) implies continuity of
%	$\Fun{a}$ and $F[\overline a]$. Hence for $N\geq N_0(\delta)$ we obtain
%	\begin{align*}
%		\Biggl|
%			&I-\biggl\|\int_{K_\varepsilon^t}\bigl(F[\overline a]-\Fun{a}\bigr)(x-y)d\mu_t(y)\biggr\|\Biggr|\\
%			&\leq\biggl\|\int_{K_\varepsilon^t}\bigl(F[a_N]-\Fun{a}\bigr)(x-y)d\mu_N_t(y)
%					-\int_{K_\varepsilon^t}\bigl(F[\overline a]-\Fun{a}\bigr)(x-y)d\mu_t(y)\biggr\|\\
%			&\leq\Biggl\|\int_{K_\varepsilon^t}
%				\bigl(F[a_N]-F[\overline a]\bigr)(x-y)d\mu_t^N(y)\Biggr\|\\
%			&\qquad +\biggl\|\int_{K_\varepsilon^t}\bigl(F[\overline a]-\Fun{a}\bigr)(x-y)d\mu_N_t(y)
%					-\int_{K_\varepsilon^t}\bigl(F[\overline a]-\Fun{a}\bigr)(x-y)d\mu_t(y)\biggr\|\\
%			&\leq\int_{K_\varepsilon^t}\frac{\delta}{2}d\mu_t^N(y)+\frac{\delta}{2}\leq\delta\,,
%	\end{align*}
%	which implies
%	\begin{equation*}
%		\lim_{N\rightarrow\infty}I_N
%			=\Biggl\|\int_{K_\varepsilon^t}\bigl(F[\overline a]-\Fun{a}\bigr)(x-y)d\mu_t(y)\Biggr\|\,.
%	\end{equation*}
%	For the term $II_N$ we first obtain
%	\begin{equation*}
%		II_N\leq 2\|a\|_\infty\int_{A\setminus K_\varepsilon^t}d\mu_t^N(y)
%			\leq 2\|a\|_\infty\int_{\R^d}\psi_\varepsilon(y)d\mu_t^N(y)\,,
%	\end{equation*}
%	where $\psi_\varepsilon$ is a bounded continuous (bump) function approximating
%	$\chi_{A\setminus K_\varepsilon^t}$ in $L_1(\mu_t)$ from above, i.e.
%	$\psi_\varepsilon\geq\chi_{A\setminus K_\varepsilon^t}$ and
%	\begin{equation*}
%		\|\psi_\varepsilon-\chi_{A\setminus K_\varepsilon^t}\|_{L_1(\mu_t)}\leq\varepsilon\,.
%	\end{equation*}
%	But then the weak convergence of $\mu_t^N$ implies
%	\begin{equation*}
%		\int_{\R^d}\psi_\varepsilon(y)d\mu_t^N(y)
%			\longrightarrow\int_{\R^d}\psi_\varepsilon(y)d\mu_t(y)\leq 2\varepsilon
%	\end{equation*}
%	by choice of $\psi_\varepsilon$ and $K_\varepsilon^t$. Put together, we thus have
%	\begin{equation*}
%		\limsup_{N\rightarrow\infty}II_N\leq 4\|a\|_\infty\varepsilon\,.
%	\end{equation*}
%	In turn, this yields
%	\begin{equation*}
%		\liminf_N\Biggl\|\int_{\R^d}\bigl(F[a_N]-\Fun{a}\bigr)(x-y)d\mu_t^N(y)\Biggr\|
%			\geq\Biggl\|\int_{K_\varepsilon^t}\bigl(F[\overline a]-\Fun{a}\bigr)(x-y)d\mu_t(y)\Biggr\|
%				-4\|a\|_\infty\varepsilon\,,
%	\end{equation*}
%	so far for every choice of $\varepsilon>0$ and $t>0$. Integrating over $t$ then yields \eqref{eq-lower-semi-2}.
%\end{proof}
%
%
%\begin{lemma}\label{lemma-semicontinuous-2}
%	Let $(a_N)_N\subset L_2(\R_+,\rho)$ be a sequence of continuous, weakly differentiable functions such that for all
%	$N$
%	\[
%		\|a_N\|_{W^1_1(\R_+,\rho)}\leq C_0\,,
%	\]
%	and let $(a_N)_N$ be uniformly bounded at a point.
%	
%	Then there exists a subsequence $(a_{N_k})_k$ which converges $\cl_1$-a.e.
%\end{lemma}
%
%\begin{proof}
%	If we denote by $\phi_t$ the Radon-Nicodym derivative of $\varrho_t$ w.r.t. $\cl_1$, for arbitrary fixed
%	$t\in [0,T]$ we can apply Helly's Selection Theorem to the functions $(a_N\phi_t)_N$. Then there exists a
%	subsequence $(a_{N_k}\phi_t)_k$ which converges pointwise $\cl_1$-a.e. to some
%	function $\overline a\phi_t$ of bounded variation.
%\end{proof}

\begin{proof}[\normalfont\bf Proof of Theorem \ref{thm}]
%	By the respective construction of the functionals $E_N$, the sequence of minimizers
%	$(\widehat a_N)_N$ satisfies the assumptions of one of the above lemmas. Hence it has a pointwise $\rho$-a.e. or
%	uniformly, respectively, convergent subsequence with limit function $\overline a$. We wish to show that
%	$\overline a$ is a minimizer of $E$, then by \eqref{eq-coercive} we have $\overline a=a$. Since in this way for
%	every subsequence we can extract a subsubsequence converging to the same limit function $a$, we can infer that
%	the entire sequence converges to $a$. For simplicity we will only treat the case $p=\infty$; the case $1<p<\infty$
%	can be reduced to the proof for $p=1$ in view of $W^1_p\hookrightarrow W^1_1$; finally, the case $p=1$ uses the
%	same arguments as $p=\infty$, with Lemma \ref{lemma-semicontinuous-1} replaced by Lemmas
%	\ref{lemma-semicontinuous-2a} and \ref{lemma-semicontinuous-2}.
	The sequence of minimizers $(\widehat a_N)_{N \in \N}$ is by definition a subset of $X_{M,K}$, hence by Proposition \ref{XMdef} it admits a subsequence $(\widehat a_{N_k})_{k \in \N}$ uniformly converging to a function $\widehat a \in X_{M,K}$.
	
	To show the optimality of $\widehat a$ in $X_{M,K}$, let $b\in X_{M,K}$ be given. By Definition \ref{VNdef}, we can find a sequence $(b_N)_{N \in \N}$ converging uniformly to $b$ on $K$ such that $b_N\in V_N$ for every $N\in \N$. Hence, by Lemma \ref{lemma-semicontinuous-1}, we have
	\begin{align*}
		\mathcal E(b)=\lim_{N\rightarrow\infty} \mathcal E_{N}(b_{N}).
	\end{align*}
	Now, by the optimality of $\widehat a_{N_k}$ in $V_N$ and again by Lemma \ref{lemma-semicontinuous-1}, it follows that
	\begin{align*}
		\mathcal E(b)=\lim_{N\rightarrow\infty}\mathcal E_N(b_N)
			= \lim_{k \rightarrow\infty}\mathcal E_{N_k}(b_{N_k})
			\geq\lim_{k \rightarrow\infty}\mathcal E_{N_k}(\widehat a_{N_k})
			= \mathcal E(\widehat a)\,.
	\end{align*}
	We can therefore conclude that for every $b \in X_{M,K}$
	\begin{align}\label{fond}
		\mathcal E(b)\geq \mathcal E(\widehat a)\,.
	\end{align}
 In particular, \eqref{fond} applies to $b=a\in X_{M,K}$ (by the particular choice of $M$), which finally implies
	\begin{align*}
		0=\mathcal E(a)\geq \mathcal E(\widehat a)\geq 0\Longrightarrow \mathcal  E(\widehat a)=0,
	\end{align*}
	showing that $\widehat a$ is also a minimizer of $\mathcal E$. When the coercivity condition \eqref{eq-coercive} holds, by Proposition \ref{uniquemin} it follows $\widehat a=a$ in  $L_2(\R_+,\rho)$.
\end{proof}